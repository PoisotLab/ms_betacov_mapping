%!TEX TS-program = xelatex
\documentclass[11pt]{article}

\usepackage[english]{babel}

\usepackage{amsmath,amssymb,amsfonts}
\usepackage[utf8]{inputenc}
\usepackage[T1]{fontenc}
\usepackage{stix2}
\usepackage[scaled]{helvet}
\usepackage[scaled]{inconsolata}

\usepackage{lastpage}

\usepackage{setspace}

\usepackage{ccicons}

\usepackage[hang,flushmargin]{footmisc}

\usepackage{geometry}

\setlength{\parindent}{0pt}
\setlength{\parskip}{6pt plus 2pt minus 1pt}

\usepackage{fancyhdr}
\renewcommand{\headrulewidth}{0pt}\providecommand{\tightlist}{%
  \setlength{\itemsep}{0pt}\setlength{\parskip}{0pt}}

\makeatletter
\newcounter{tableno}
\newenvironment{tablenos:no-prefix-table-caption}{
  \caption@ifcompatibility{}{
    \let\oldthetable\thetable
    \let\oldtheHtable\theHtable
    \renewcommand{\thetable}{tableno:\thetableno}
    \renewcommand{\theHtable}{tableno:\thetableno}
    \stepcounter{tableno}
    \captionsetup{labelformat=empty}
  }
}{
  \caption@ifcompatibility{}{
    \captionsetup{labelformat=default}
    \let\thetable\oldthetable
    \let\theHtable\oldtheHtable
    \addtocounter{table}{-1}
  }
}
\makeatother

\usepackage{array}
\newcommand{\PreserveBackslash}[1]{\let\temp=\\#1\let\\=\temp}
\let\PBS=\PreserveBackslash

\usepackage[breaklinks=true]{hyperref}
\hypersetup{colorlinks,%
citecolor=blue,%
filecolor=blue,%
linkcolor=blue,%
urlcolor=blue}
\usepackage{url}

\usepackage{caption}
\setcounter{secnumdepth}{0}
\usepackage{cleveref}

\usepackage{graphicx}
\makeatletter
\def\maxwidth{\ifdim\Gin@nat@width>\linewidth\linewidth
\else\Gin@nat@width\fi}
\makeatother
\let\Oldincludegraphics\includegraphics
\renewcommand{\includegraphics}[1]{\Oldincludegraphics[width=\maxwidth]{#1}}

\usepackage{longtable}
\usepackage{booktabs}

\usepackage{color}
\usepackage{fancyvrb}
\newcommand{\VerbBar}{|}
\newcommand{\VERB}{\Verb[commandchars=\\\{\}]}
\DefineVerbatimEnvironment{Highlighting}{Verbatim}{commandchars=\\\{\}}
% Add ',fontsize=\small' for more characters per line
\usepackage{framed}
\definecolor{shadecolor}{RGB}{248,248,248}
\newenvironment{Shaded}{\begin{snugshade}}{\end{snugshade}}
\newcommand{\KeywordTok}[1]{\textcolor[rgb]{0.13,0.29,0.53}{\textbf{#1}}}
\newcommand{\DataTypeTok}[1]{\textcolor[rgb]{0.13,0.29,0.53}{#1}}
\newcommand{\DecValTok}[1]{\textcolor[rgb]{0.00,0.00,0.81}{#1}}
\newcommand{\BaseNTok}[1]{\textcolor[rgb]{0.00,0.00,0.81}{#1}}
\newcommand{\FloatTok}[1]{\textcolor[rgb]{0.00,0.00,0.81}{#1}}
\newcommand{\ConstantTok}[1]{\textcolor[rgb]{0.00,0.00,0.00}{#1}}
\newcommand{\CharTok}[1]{\textcolor[rgb]{0.31,0.60,0.02}{#1}}
\newcommand{\SpecialCharTok}[1]{\textcolor[rgb]{0.00,0.00,0.00}{#1}}
\newcommand{\StringTok}[1]{\textcolor[rgb]{0.31,0.60,0.02}{#1}}
\newcommand{\VerbatimStringTok}[1]{\textcolor[rgb]{0.31,0.60,0.02}{#1}}
\newcommand{\SpecialStringTok}[1]{\textcolor[rgb]{0.31,0.60,0.02}{#1}}
\newcommand{\ImportTok}[1]{#1}
\newcommand{\CommentTok}[1]{\textcolor[rgb]{0.56,0.35,0.01}{\textit{#1}}}
\newcommand{\DocumentationTok}[1]{\textcolor[rgb]{0.56,0.35,0.01}{\textbf{\textit{#1}}}}
\newcommand{\AnnotationTok}[1]{\textcolor[rgb]{0.56,0.35,0.01}{\textbf{\textit{#1}}}}
\newcommand{\CommentVarTok}[1]{\textcolor[rgb]{0.56,0.35,0.01}{\textbf{\textit{#1}}}}
\newcommand{\OtherTok}[1]{\textcolor[rgb]{0.56,0.35,0.01}{#1}}
\newcommand{\FunctionTok}[1]{\textcolor[rgb]{0.00,0.00,0.00}{#1}}
\newcommand{\VariableTok}[1]{\textcolor[rgb]{0.00,0.00,0.00}{#1}}
\newcommand{\ControlFlowTok}[1]{\textcolor[rgb]{0.13,0.29,0.53}{\textbf{#1}}}
\newcommand{\OperatorTok}[1]{\textcolor[rgb]{0.81,0.36,0.00}{\textbf{#1}}}
\newcommand{\BuiltInTok}[1]{#1}
\newcommand{\ExtensionTok}[1]{#1}
\newcommand{\PreprocessorTok}[1]{\textcolor[rgb]{0.56,0.35,0.01}{\textit{#1}}}
\newcommand{\AttributeTok}[1]{\textcolor[rgb]{0.77,0.63,0.00}{#1}}
\newcommand{\RegionMarkerTok}[1]{#1}
\newcommand{\InformationTok}[1]{\textcolor[rgb]{0.56,0.35,0.01}{\textbf{\textit{#1}}}}
\newcommand{\WarningTok}[1]{\textcolor[rgb]{0.56,0.35,0.01}{\textbf{\textit{#1}}}}
\newcommand{\AlertTok}[1]{\textcolor[rgb]{0.94,0.16,0.16}{#1}}
\newcommand{\ErrorTok}[1]{\textcolor[rgb]{0.64,0.00,0.00}{\textbf{#1}}}
\newcommand{\NormalTok}[1]{#1}

\newlength{\cslhangindent}
\setlength{\cslhangindent}{1.5em}
\newlength{\csllabelwidth}
\setlength{\csllabelwidth}{3em}
\newenvironment{CSLReferences}[3] % #1 hanging-ident, #2 entry spacing
 {% don't indent paragraphs
  \setlength{\parindent}{0pt}
  % turn on hanging indent if param 1 is 1
  \ifodd #1 \everypar{\setlength{\hangindent}{\cslhangindent}}\ignorespaces\fi
  % set entry spacing
  \ifnum #2 > 0
  \setlength{\parskip}{#2\baselineskip}
  \fi
 }%
 {}
\usepackage{calc} % for \widthof, \maxof
\newcommand{\CSLBlock}[1]{#1\hfill\break}
\newcommand{\CSLLeftMargin}[1]{\parbox[t]{\maxof{\widthof{#1}}{\csllabelwidth}}{#1}}
\newcommand{\CSLRightInline}[1]{\parbox[t]{\linewidth}{#1}}
\newcommand{\CSLIndent}[1]{\hspace{\cslhangindent}#1}\geometry{verbose,letterpaper,tmargin=2.2cm,bmargin=2.2cm,lmargin=2.2cm,rmargin=2.2cm}

\usepackage{lineno}
\usepackage[nolists,noheads]{endfloat}

\pagestyle{plain}

\tolerance=1
\emergencystretch=\maxdimen
\hyphenpenalty=10000
\hbadness=10000

\doublespacing

\fancypagestyle{normal}
{
  \fancyhf{}
  \fancyfoot[R]{\footnotesize\sffamily\thepage\ of \pageref*{LastPage}}
}
\begin{document}
\raggedright
\thispagestyle{empty}
{\Large\bfseries\sffamily The coevolutionary mosaic of
bat-betacoronaviruses spillover risk}
\vskip 5em

%
Norma\,Forero Rocio Munoz%
%
\,\textsuperscript{1,2}\quad %
Renata L.\,Muylaert%
%
\,\textsuperscript{3}\quad %
Stephanie N.\,Seifert%
%
\,\textsuperscript{4}\quad %
Gregory F.\,Albery%
%
\,\textsuperscript{5}\quad %
Colin J.\,Carlson%
%
\,\textsuperscript{5}\quad %
\href{https://orcid.org/0000-0002-0735-5184}{Timothée\,Poisot}%
%
\,\textsuperscript{1,2,‡}

\textsuperscript{1}\,Université de
Montréal\quad \textsuperscript{2}\,Québec Centre for Biodiversity
Sciences\quad \textsuperscript{3}\,Molecular Epidemiology and Public
Health Laboratory, Hopkirk Research Institute, Massey University,
Palmerston North, New Zealand\quad \textsuperscript{4}\,Paul G. Allen
School for Global Health, Washington State University, Pullman, WA,
United States\quad \textsuperscript{5}\,???

\textsuperscript{‡}\,These authors contributed equally to the work\\

\textbf{Correspondance to:}\\
Timothée Poisot --- \texttt{timothee.poisot@umontreal.ca}\\

\vfill
This work is released by its authors under a CC-BY 4.0 license\hfill\ccby\\
Last revision: \emph{\today}

\clearpage
\thispagestyle{empty}

\vfill


        {\bfseries Purpose:}\,This template provides a series of scripts
to render a markdown document into an interactive website and a series
of PDFs.\\%
        {\bfseries Motivation:}\,It makes collaborating on text with
GitHub easier, and means that we never need to think about the
output.\\%
        {\bfseries Internals:}\,GitHub actions and a series of python
scritpts. The markdown is handled with \texttt{pandoc}.\\%
    
\vfill

\clearpage
\linenumbers
\pagestyle{normal}

Spillover risk is not unidimensional. From the standpoint of an animal
community, \emph{i.e.} a pool of suitable hosts, it is driven by a
multiplicity of factors (Plowright et al. 2017). The global richness of
hosts is one such component commonly mentioned/analysed (see \emph{e.g.}
Anthony et al. 2017 for coronaviruses), but there is an argument to be
made that species who are not competent (or know) hosts of a specific
virus genus may not factor into this (Plowright et al. 2015), or that
species who are assumed to share viruses at different rates should be
weighted accordingly (Albery et al. 2020). In mammals, key functional
traits (for which phylogeny is a reasonable proxy) are determinants of
the spillover potential (Olival et al. 2017); these include, notably,
body mass and affinity for urban environments (Albery et al. 2022).
Finally, especially when the pool of potential hosts spans the entire
globe, there may be local host pools that are highly unique; not having
been observed in other locations, these can act on the overall risk
either by providing novel contact opportunities, reflecting unique
host-environment combinations (Engering et al. 2013), or facilitating
rapid evolutionary changes in specialism of their pathogens (Agosta et
al. 2010). In the specific case of generalist pathogens, there is
conceptual and empirical support to the idea that these community- level
mechanisms are even more important in driving the overall risk (Power
and Mitchell 2004).

Bats are important reservoir hosts for different classes of
microorganisms (Chu 2008, Donaldson 2010, Li 2010), some of which can
threaten human health. Chiropterans emerged around 64 million years ago
and are one of the most diverse mammalian orders, with an estimated
richness of more than 12000 species, (Peixoto F et al, 2018) and 14325
known species (Simmons \& Cirranello). They exhibit a broad variety of
habitat use, behaviour, and feeding strategies, resulting in their
playing an essential role in the delivery of several ecosystem services
tied to important ecosystem-derived benefits (Kasso 2013). For example,
over two-thirds of bats are know to be either obligate or facultative
insectivorous mammals, therefore playing an important role in the
regulation of insect pests that can affect crops (Williams-Guillen
2011), and vectors of diseases that put a risk on human health
(Gonsalves 2013). Because bats are globally distributed and have a long
evolutionary history, phylogeographic and biogeographic approaches are
required to shed light on the extant distribution of coevolutionary
processes between bats and the pathogens they carry. Not all areas in
which bats, viruses, and human are co-occuring are facing a risk of
spillover towards human populations, and the areas in which this risk
exist may not be facing risks of the same nature and magnitude.

In this paper, we examine the biogeographic structure of
bats-betacoronaviruses associations, based on a curated dataset of known
and recently discovered hosts. This work is important both as a
description of the bats-betacoronavirus complex, but aslo because more
broadly, bats are known reservoirs for a variety of emerging viruses
(Calisher 2006), making balancing the needs for bat conservation and
disease prevention a potentially difficult act and a source of
human-wildlife conflicst, especially in more densely populated areas
(REF). By drawing on concepts from the Geographic Mosaic Theory of
Coevolution (REF), we turn these associations into a spatially explicit
additive mapping of zoonotic risk components, which reveals extreme
heterogeneity of risk at the global scale; furthermore, we identify the
Amazon and South-Eastern Asia as hotspots of phylogenetic
distinctiveness of betacoronaviruses; surprisingly, current data suggest
that viral sharing between hosts in high in the Amazon and low in
South-Eastern Asia, which has the potential to result in different
evolutionary dynamics between these two regions.

\hypertarget{methods}{%
\section{Methods}\label{methods}}

\hypertarget{known-betacoronavirus-hosts}{%
\subsection{Known betacoronavirus
hosts}\label{known-betacoronavirus-hosts}}

We downloaded the data on bats hosts of betacoronaviruses assembled by
Becker et al. (2022) from
\texttt{https://www.viralemergence.org/betacov} on Aug.~2021, and
filtered it to ``known'' hosts (established before the emergence of
SARS-CoV-2) and ``novel'' hosts (confirmed through sampling since the
emergence of SARS-CoV-2). This database was assembled by a combination
of data mining and literature surveys.

\hypertarget{bats-occurrences}{%
\subsection{Bats occurrences}\label{bats-occurrences}}

We downloaded the rangemap of every extant bat species that was either
classified as an empirically documented or a suspected host of
beta-coronaviruses (Becker et al.~2020), according to recent IUCN data
(IUCN 2021). The range maps were subsequently rasterized at a resolution
of approximately TK TP. For every pixel in the resulting raster where at
least one bat host of betacoronavirus was present, we extract the
species pool, which was used to calculate the following risk assessment
components: phylogenetic diversity, bat compositional uniqueness, and
predicted viral sharing risk.

\hypertarget{bats-phylogeography}{%
\subsection{Bats phylogeography}\label{bats-phylogeography}}

For every pixel, we measured Faith's Phylogenetic Diversity (Faith 1992)
based on a recent synthetic tree with robust time calibration, covering
about 6000 mammalian species (Upham et al.~2019). Faith's PD measures
the sum of unique branches from an arbitrary root to a set of tips, and
comparatively larger values indicate a more phylogenetic diverse species
pool. We measured phylogenetic diversity starting from the root of the
entire tree (and not from Chiroptera); this bears no consequences on the
resulting values, since all branches leading up to Chiroptera are only
counted one per species pool, and (as we explain when describing the
assembly of the composite risk map), all individual risk components are
ranged in {[}0,1{]}. This measure incorporates a richness component,
which we chose not to correct for; the interpretation of the
phylogenetic diversity is therefore a weighted species richness, that
accounts for phylogenetic over/under-dispersal in some places.

\hypertarget{bats-compositional-uniqueness}{%
\subsection{Bats compositional
uniqueness}\label{bats-compositional-uniqueness}}

For every species pool, we measured its Local Contribution to
Beta-Diversity (Legendre and De Cáceres 2013); LCBD works from a
species-data matrix (traditionally noted as Y), where species are rows
and sites are columns, and a value of 1 indicates occurrence. We
extracted the Y matrix assuming that every pixel represents a unique
location, and following best practices (Legendre and Condit 2019)
transformed it using Hellinger's distance to account for unequal bat
richness at different pixels. The correction of raw community data is
particularly important for two reasons: first, it prevents the artifact
of richer sites having higher importance; second, it removes the effect
of overall species richness, which is already incorporated in the
phylogenetic diversity component. High values of LCBD indicate that the
pixel has a community that is on average more dissimilar in species
composition than what is expected knowing the entire matrix, i.e.~a more
unique community.

\hypertarget{viral-sharing-between-hosts}{%
\subsection{Viral sharing between
hosts}\label{viral-sharing-between-hosts}}

For all bat hosts of betacoronaviruses, we extracted their predicted
viral sharing network (Albery et al.~2020). This network stores pairwise
values of viral community similarity. To project viral sharing values
into a single value for every pixel, we averaged the pairwise scores.
High values of the average sharing propensity means that this specific
extant bat assemblage is likely to be proficient at exchanging viruses.

\hypertarget{composite-risk-map}{%
\subsection{Composite risk map}\label{composite-risk-map}}

To visualize the aggregated risk at the global scale, we combine the
three individual risk components (phylogenetic diversity, compositional
uniqueness, and viral sharing) using an additive color model (Seekell et
al.~2018). In this approach, every risk component gets assigned a
component in the RGB color model (phylogenetic diversity is green,
compositional uniqueness is red, and viral sharing is blue). In order to
achieve a valid RGB measure, all components are re-scaled to the
{[}0,1{]} interval. This additive model conveys both the intensity of
the overall risk, but also the nature of the risk as colors diverge
towards combinations of values for three risk components.

\hypertarget{viral-phylogeography}{%
\subsection{Viral phylogeography}\label{viral-phylogeography}}

We used the following query to pull all betacoronavirus sequence data
from the GenBank Nucleotide database except SARS-CoV-2;
(``Betacoronavirus''{[}Organism{]} OR betacoronavirus{[}All Fields{]})
NOT (``Severe acute respiratory syndrome coronavirus 2''{[}Organism{]}
OR sars-cov-2{[}All Fields{]}). We added a single representative
sequence for SARS-CoV-2 and manually curated to remove sequences without
the RNA-dependent RNA polymerase (RdRp) sequence or that contained words
indicating recombinant or laboratory strains including ``patent,''
``mutant,'' ``GFP,'' and ``recombinant.'' We filtered over-represented
taxa including betacoronavirus 1, hCoV-OC43, Middle East respiratory
syndrome coronavirus, Murine hepatitis virus, and hCoV-HKU1. Curated
betacoronavirus RdRp sequences were then aligned using MAFFT v 1.4.0
(Katoh and Standley 2013, Supplemental X) and a maximum likelihood tree
reconstructed in IQ-TREE v 1.6.12 (Nguyen et al.~2015) with ModelFinder
(Kalyaanamoorthy et al.~2017) ultrafast bootstrap approximation (Hoang
et al. 2018) and the following parameters (STEPH WILL ADD, Supplemental
X).

\hypertarget{viral-evolutionary-diversification}{%
\subsection{Viral evolutionary
diversification}\label{viral-evolutionary-diversification}}

We first tested the hypothesis that hotspots of viral diversification
would track hotspots of bat diversification. To do so, we plotted the
number of known bat hosts (specifically only those included in the
phylogeny, so there was a 1:1 correspondence between data sources)
against the ``mean evolutionary distinctiveness'' of the associated
viruses. To calculate this, we derived the fair proportions evolutionary
distinctiveness (Isaac et al., 2007) for each of the viruses in the
tree, then averaged these at the bat species level, projected these
values onto their geographic distributions, and averaged across every
bat found in a given pixel. As such, this can be thought of as a map of
the mean evolutionary distinctiveness of the known viral community
believed to be associated with a particular subset of bats present.

\hypertarget{co-distribution-of-hosts-and-viral-hotspots}{%
\subsection{Co-distribution of hosts and viral
hotspots}\label{co-distribution-of-hosts-and-viral-hotspots}}

Subsequently, we tested the hypothesis that the biogeography of bat
betacoronaviruses should track the biogeography of their hosts. To test
this idea, we loosely adapted a method from Kreft \& Jetz (2010), who
proposed a phylogenetic method for the delineation of animal
biogeographic regions. In their original method, a distance matrix -
where each row or column represents a geographic raster's grid cell, and
the dissimilarity values are the ``beta diversity similarity'' of their
community assemble - undergoes non-metric multidimensional scaling
(NMDS); the first two axes of the NMDS are projected geographically
using a four-color bivariate map.

Here, we build on this idea with an entirely novel methodology. First,
we measure the phylogenetic distance between the different viruses in
the betacoronavirus tree by using the cophenetic function in `ape';
subsequently, we take a principal components analysis of that distance
matrix (readily interchangeable for NMDS in this case) to project the
viral tree into an n-dimensional space. We then take the first two
principal components and, as with the evolutionary distinctiveness
analysis, aggregated these to a mean host value and projected them using
a four-color bivariate map.

\hypertarget{outbreaks-data-geo-referencing}{%
\subsection{Outbreaks data
geo-referencing}\label{outbreaks-data-geo-referencing}}

Finally, we provide a summary visualization of what available
information describes the spillover of zoonotic betacoronaviruses of bat
origin where data was available before and up through the COVID-19
pandemic. The SARS-CoV-2 outbreak was georeferenced to the initial case
cluster in Wuhan, China; SARS-CoV was georeferenced based on the cave
with the closest known viruses circulating in nature (Hu et al.~2017
PLoS Pathogens), and a nearby location where serological (antibody)
evidence has indicated human exposure to SARS-like viruses (Wang et
al.~2018 Virologica Sinica). For MERS-CoV, we presented the index cases
available from a recently-published compendium of MERS-CoV cases
(Ramshaw et al.~2019); these are largely if not all presumed to be
camel-to-human transmission, and the precise origin point of MERS-CoV in
bats is uncertain. Not shown is a recent case of a recombinant canine
coronavirus that showed the ability to infect humans, both because this
study was published after the beginning of the COVID-19 pandemic and
because bats' involvement in this cycle of transmission has been
marginal to non-existent.

\hypertarget{results}{%
\section{Results}\label{results}}

\hypertarget{host-distribution}{%
\subsection{Host distribution}\label{host-distribution}}

Chiroptera are an hyperdiverse group, distributed in a large part if the
world, and are an important reservoir for different strains of
betacoronaviruses (Drexler et al., 2014); this has attracted attention
to areas where high diversity of bats can be an important issue for
human health (Calisher et al., 2006). Accordingly, we collected the IUCN
rangemaps for known hosts of betacoronaviruses, to illustrate where
hotspots of host diversity are. These results are presented in Fig xx.a.
As per our current knowledge of which bats are hosts of
betacoronaviruses, these hotspots are primarily South-East Asia, parts
of Europe, and to a lesser extent sub-saharan Africa. Even the subset of
chiroptera that are hosts of betacoronaviruses fits the evolutionary
timeline of the group. Chiropterans can be classified as Microchiroptera
and macrochiroptera, where macrochiroptera have an older history from an
evolutionary perspective compared to macrochiroptera (Springer, 2013;
Teeling et al., 2005). South-East Asia has a high diversity of bats
(Kingston, 2010), and our results show that part of that diversity
includes betacoronavirus hosts. High density of hosts sharing the same
virus (albeit possibly different strains) calls into question the
evolution of the bat antiviral immune system and its co-evolution with
viruses, which may result in distinct immunological responses in
different area, as evidenced in other bat species (Banerjee et al.,
2020)

\hypertarget{viral-evolutionary-distinctiveness}{%
\subsection{Viral evolutionary
distinctiveness}\label{viral-evolutionary-distinctiveness}}

Higher host diversity may not result in a higher viral diversity; for
this reason, we quantified and mapped the evolutionary distinctiveness
of betacoronaviruses, based on \ldots. Viral evolutionary
distinctiveness largely tracks host diversity, particularly in southern
China but oddly not throughout the rest of southeast Asia, perhaps
indicating that many distinctive viruses remain to be discovered in this
region (an idea that is unsurprising given the growing realization,
around the emergence of SARS-CoV-2, that a unique lineage of similar
viruses are widespread in bats but still mostly undescribed). The most
distinct betacoronaviruses are found in South America, a region with a
comparatively lower number of hosts; this suggests that the South
American bat-betacoronvirus complex has been more isolated, and is
probably undergoing a different co-evolutionary dynamic. Alternatively,
this distinctiveness hostpot may be a product of under-sampling:
South-America is one of the places where the fewest betacoronaviruses
have been discovered (Anthony et al., 2017), and adding more viruses
would bring the distinctiveness of known sequences down. Previous work
has suggested the Americas may be a hotspot of both undiscovered bat
viruses in general (Olival) and coronavirus specifically (Anthony),
though not necessarily betacoronaviruses, and particularly not those in
clades with notable zoonotic potential (c.f. Anthony).

\hypertarget{geographic-mosaic-of-bat-betacoronavirus-risk}{%
\subsection{Geographic Mosaic of bat-betacoronavirus
risk}\label{geographic-mosaic-of-bat-betacoronavirus-risk}}

In order to turn the hypotheses based on the Geographic Mosaic Theory of
Coevolution into a measure of risk, we overlapped three components:
viral sharing, i.e.~the chance that two bats will share viruses overall;
Local Contribution to Beta Diversity, i.e.~the fact that a bat community
is compositionally unique compared to the average compositional
similarity across the entire system; finally, the phylogenetic
diversity, i.e.~how dispersed the bats in a location are within the tree
of life. These results are presented using an additive color mapping in
Figure xx, and lead to the definition of broad biogeographic regions of
risk, where the same color represents the same type of risk. Pairwise
maps of the three components are present in supplementary materials.

From the perspective of spillover risk, the most important combination
of factors is a high phylogenetic diversity of hosts with low viral
sharing; this, essentially, means that very different betacoronavirus
could co-exist within the same place. This is particularly the case
given that betacoronaviruses often evolve and even achieve host shifts
through recombination, which requires the co-occurrence of sufficiently
distinct viruses to be a major driver of emergence. In Fig. xx, this
corresponds to yellow to pale green areas, which are essentially limited
to South-Eastern Asia, and to some part of Sub-Saharan Africa. Adopting
a geographic mosaic theory perspective on risk, other regions of the
world are of lesser concern.

Available data on bat betacoronavirus spillover into humans (TP overlay
on the figure) is limited and circumstantial at best for these purposes,
but our risk maps suggest that the areas predicted by prior expectations
about host biogeography correspond loosely to those where previous
emergence events have been recorded. Areas with high bat diversity and
high turnover may facilitate the evolutionary radiation of viruses,
matching previous findings that the diversification of bat coronaviruses
is driven largely by host shifts (inter-genus or higher levels of
cross-species transmission) and, to a lesser degree, cospeciation and
sharing (intra-genus cross-species transmission; Anthony et al.~2017).
This diversification - while not an actual risk factor for spillover
itself - likely increases the random chance of a virus with the raw
genomic components required for the potential to infect humans.

\hypertarget{global-distribution-of-spillover-risk}{%
\subsection{Global distribution of spillover
risk}\label{global-distribution-of-spillover-risk}}

Based on the previous result, we extracted the yellow component of the
risk map (TP add methods), to provide a single measure of risk varying
between 0 and 1. This measure is presented in Fig. xxA. However, this
maps the potential risk, which must be weighed by the potential for
contacts with humans. As a proxy for this measure, we used the
proportion of build/urban land from the EarthEnv dataset: this is a
reasonable proxy for the density of humans per unit area, which
increases the probability of pathogen spread more widely (Hazarie et
al., 2021). Since human activity is required to amplify the frequency of
virus encounters and thus create areas of viral amplification, mapping
the potential risk against measures of land use is required to generate
a more actionable assessment of risk. This map is presented in Fig. xxB.
Most of South America and Europe are at low risk, as although densely
populated, settlements tend to be in areas with lower potential risk.
However, this mapping reveals that South-East Asia, the Indian
subcontinent, and parts of sub-Saharan Africa, are at high risk due to
the overlap between built areas and bat communities representing more
opportunities for cross-species transmission of betacoronaviruses.

\hypertarget{discussion}{%
\section{Discussion}\label{discussion}}

Driven by the need to understand the ecological factors involved in the
emergence of viral pathogens, we spatially mapped bat-betacoronavirus
interactions worldwide, using (i) a database of known betacov
hosts(Becker et al., 2020), and (ii) range maps for the hosts according
to IUCN (IUCN 2021). To reflect the fact that the risk posed by viruses
has many ecological origins, we quantified the phylogenetic diversity of
hosts, their compositional uniqueness, and the expected viral sharing.
Because these components of risk matter when contrasted to human
density, we compared them to a proxy, namely the proportion of each
pixel that is covered by urban or built land. This provides a synthetic
risk map, allowing to identifying of hotspots where the
bat-betacoronavirus system may originate viruses in humans. SE Asia is
one of the regions with the highest risk since, according to our
results, several of its conditions could increase the risk of
transmission of the virus.

Bats are found worldwide and are one of the most diverse groups among
mammals (Moratelli \& Calisher, 2015). Previous research (Anthony et
al., 2017; Mollentze \& Streicker, 2020) states that locally diverse bat
communities could maintain more viruses and hence, a higher probability
of having a pathogen that could represent a risk for human health. This
probability involves multiple factors, among which the relatedness of
hosts (which can make the jumps easier (Longdon et al., 2011; Mollentze
et al., 2020; Wolfe et al., 2007), and the overall tendency of hosts
within a locality to share viruses, which may limit viral diversity
because of within-host competition (Leeks et al., 2018; Sallinen et al.,
2020). Species richness, therefore, is not a sufficient measure of viral
risk. This is exemplified in our results, where both South America and
South-Eastern Asia have a high species richness of betacov hosts, but
only the latter region has a high risk. Specifically, because previous
studies propose that Asia is important when it comes to understanding
the evolutionary origin of various mammalian taxa (Beard C K, 1988).
Including bats (Yu et al., 2014), which could support the relationship
between evolutionary time and the development of an immune system with
characteristics that allow them to be better adapted to infection by
emerging viruses (Gorbunova et al., 2020; Irving et al., 2021) may be
related to a wide variety of diets (Jones et al., 2022; Moreno Santillán
et al., 2021; Banerjee et al., 2020; Schneeberger et al., 2013).

Our study focuses largely on the biogeography of hosts. Yet, we know
that viruses with high host plasticity, that is, the ability of a given
virus to adapt to various taxonomic orders and ecological groups
(Kreuder Johnson et al., 2015); are more likely to amplify viral
spillover, followed by secondary human-to-human transmission, and
geographical spread (Hazarie et al., 2021). High viral host plasticity
is an especially important trait for RNA viruses such as betacov
(Kreuder Johnson et al., 2015; Haddad et al., 2021). Indeed, our
analysis of viral sequences reveals that Latin America is a hotspot of
viral distinctiveness, suggesting that this part of the bats-betacov
system may be undergoing independent evolutionary dynamics (related
species sharing viruses that are different from the rest of the global
pool). The other hotspot of viral distinctiveness is S.E. Asia, in which
richness is high but sharing is low; this suggests a different type of
evolutionary dynamics (unrelated viruses coevolving with evolutionarily
distinct hosts, generating high diversity locally).

This diversity of hosts and how the exchange of viruses occurs between
species, is largely affected by the different environmental changes, as
the case of sarbecovirus bats reservoirs (Muylaert et al., 2021) where
they are affected by the area of the cave or the alteration of the
forest, which could result in modifications of host distribution.
Additionally, our results highlight the importance of Asia as a betacov
hotspot, which is consistent with recent studies (Muylaert et al.,
2021), where projections on this area suggest that new future events of
sarbecovirus viral exchange might be easily spread among species or
humans.

There are several factors that drive changes in the diversity of bats
(Alves et al., 2018), but human activities' effects on the ecosystem
(like modifications of land use) could significantly decrease it.
Therefore, it can be suggested that changes in the diversity of betacovs
in bats are linked to their biogeographic variation, and human
population density and other anthropogenic factors are decisive
moderators for its implications in public health. With the increase of
contact between humans and potential hosts, we also increase the risk of
emergence of novel diseases (Johnson et al., 2020), as previous studies
on RNA viruses suggest the importance of host phylogeography at the time
of virus dispersal (Gryseels et al., 2017).

One of these scenarios where interaction between bats and humans can
occur can be seed dispersal in tropical agroecosystems. It opens the
discussion of whether the fruits thrown by bats not only disperse seeds
but could also be a source of indirect interaction between viruses of
bat origin and humans (Deshpande et al., 2022) . This represents a
challenge for conservation strategies and disease ecology since we have
areas with potential zoonotic viruses and bat-human interaction.
However, it must still be taken into account the quantification of real
exposure from several scenarios, where there can be directly or
indirectly bat - human interaction.

Comparing scenarios of high viral exchange vs low viral exchange, open
the discussion to consider if the best scenario is where viruses easily
adapted to multiple hosts but with low virulence or easily ignored by
the immune system of the host, or where we have viruses specialized to a
specific host, but highly virulent when invade a new host. Accordingly,
the understanding of viral-host interactions from a taxonomic and
phylogenetic contributes to improving zoonoses surveillance programs.

\textbf{Acknowledgements}: We acknowledge that this study was conducted
on land within the traditional unceded territory of the Saint Lawrence
Iroquoian, Anishinabewaki, Mohawk, Huron-Wendat, and Omàmiwininiwak
nations. This work was supported by funding to the Viral Emergence
Research Initiative (VERENA) consortium including NSF BII 2021909 and a
grant from Institut de Valorisation des Données (IVADO). This research
was enabled in part by support provided by Calcul Québec
(www.calculquebec.ca) and Compute Canada (www.computecanada.ca). NF is
funded by the NSERC BIOS² CREATE program. TP and NF are funded by the
Courtois Foundation. RLM was supported by Bryce Carmine and Anne Carmine
(née Percival), through the Massey University Foundation.

\hypertarget{references}{%
\section*{References}\label{references}}
\addcontentsline{toc}{section}{References}

\hypertarget{refs}{}
\begin{CSLReferences}{1}{0}
\leavevmode\hypertarget{ref-Agosta2010HowSpe}{}%
Agosta, S. J. et al. 2010. How specialists can be generalists: Resolving
the {``parasite paradox''} and implications for emerging infectious
disease. - Zoologia (Curitiba) 27: 151--162.

\leavevmode\hypertarget{ref-Albery2020PreGlo}{}%
Albery, G. F. et al. 2020. Predicting the global mammalian viral sharing
network using phylogeography. - Nature Communications 11: 2260.

\leavevmode\hypertarget{ref-Albery2022UrbMam}{}%
Albery, G. F. et al. 2022. Urban-adapted mammal species have more known
pathogens. in press.

\leavevmode\hypertarget{ref-Anthony2017GloPat}{}%
Anthony, S. J. et al. 2017. Global patterns in coronavirus diversity. -
Virus Evolution in press.

\leavevmode\hypertarget{ref-Becker2022OptPre}{}%
Becker, D. J. et al. 2022. Optimising predictive models to prioritise
viral discovery in zoonotic reservoirs. - The Lancet Microbe in press.

\leavevmode\hypertarget{ref-Engering2013PatHos}{}%
Engering, A. et al. 2013. Pathogen--host--environment interplay and
disease emergence. - Emerging Microbes \& Infections 2: e5.

\leavevmode\hypertarget{ref-Olival2017HosVir}{}%
Olival, K. J. et al. 2017. Host and viral traits predict zoonotic
spillover from mammals. - Nature 546: 646--650.

\leavevmode\hypertarget{ref-Plowright2015EcoDyn}{}%
Plowright, R. K. et al. 2015. Ecological dynamics of emerging bat virus
spillover. - Proceedings of the Royal Society B: Biological Sciences
282: 20142124.

\leavevmode\hypertarget{ref-Plowright2017PatZoo}{}%
Plowright, R. K. et al. 2017. Pathways to zoonotic spillover. - Nature
Reviews Microbiology 15: 502--510.

\leavevmode\hypertarget{ref-Power2004PatSpi}{}%
Power, A. G. and Mitchell, C. E. 2004. Pathogen Spillover in Disease
Epidemics. - The American Naturalist 164: S79--S89.

\end{CSLReferences}

\end{document}
